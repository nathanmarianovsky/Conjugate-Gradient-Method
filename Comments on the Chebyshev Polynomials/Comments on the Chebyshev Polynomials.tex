\documentclass[12pt, letterpaper, onecolumn, conference, final]{IEEEtran}

\usepackage[margin = .5in]{geometry}
\usepackage{amsmath}
\usepackage{amsthm}
\usepackage{amssymb}
\usepackage{wasysym}
\usepackage{graphicx}
\usepackage[makeroom]{cancel}
\usepackage{polynom}
\usepackage{booktabs}

\title{Comments on the \\ Chebyshev Polynomials of the First Kind}
\author{Nathan Marianovsky}

\theoremstyle{definition}
\newtheorem{definition}{Definition}
\newtheorem{proposition}{Proposition}
\newtheorem{claim}{Claim}

\theoremstyle{plain}
\newtheorem{theorem}{Theorem}[section]
\newtheorem{example}{Example}
\newtheorem{solution}{Solution}

\renewcommand{\qedsymbol}{$\blacksquare$}
\DeclareMathOperator{\arccosh}{arccosh}

\renewcommand\thesection{\arabic{section}}

\begin{document}

\maketitle

\begin{center}
\fbox{
\begin{minipage}{7.3 in}
\begin{definition}[Trigonometric Definition] 
The \textit{Chebyshev Polynomials of the First Kind} can be thought of as the unique polynomials satisfying:
\begin{equation*}
\begin{split}
T_n(x) &= \cos(n\arccos(x)) \\
T_n(\cos(x)) &= \cos(nx)
\end{split}
\end{equation*}
where we restrict $x \in [-1,1]$ so as to stay in the domain of the inverse trigonometric function.
\end{definition}
\end{minipage}}
\end{center}

\begin{center}
\fbox{
\begin{minipage}{7.3 in}
\begin{proposition}[Hyperbolic Definition] 
One can substitute the cosine for hyperbolic cosine in the \textit{Chebyshev Polynomials of the First Kind} to arrive at an equivalent definition:
\begin{equation*}
T_n(x) = \cosh(n\arccosh(x))
\end{equation*}
\end{proposition}
\end{minipage}}
\end{center}

\begin{proof}
Remember that:
\begin{equation*}
\cosh(x) = \cos(ix) \hspace{.2cm} \text{and} \hspace{.2cm} \arccosh(x) = i\arccos(x)
\end{equation*}
Using this we have:
\begin{equation*}
\begin{split}
T_n &= \cos(n\arccos(x)) \\
&= \cos\Big( \frac{n}{i} \arccosh(x) \Big) \\
&= \cos(-in\arccosh(x)) \\
&= \cos(in\arccosh(x)) \\
&= \cosh(n\arccosh(x))
\end{split}
\end{equation*}
\end{proof}

\begin{center}
\fbox{
\begin{minipage}{7.3 in}
\begin{proposition}[Reformulation] 
The \textit{Chebyshev Polynomials of the First Kind} can be rewritten as:
\begin{equation*}
T_n(x) = \frac{1}{2} \Big( (x + \sqrt{x^2 - 1})^n + (x - \sqrt{x^2 - 1})^n \Big)
\end{equation*}
\end{proposition}
\end{minipage}}
\end{center}

\begin{proof}
We first note that:
\begin{equation*}
\cos(x) = \frac{1}{2}\Big( e^{ix} + e^{-ix} \Big) \hspace{.2cm} \text{and} \hspace{.2cm} \arccos(x) = -i\ln(x + \sqrt{x^2 - 1})
\end{equation*}
Using this we directly simplify the given definition:
\begin{equation*}
\begin{split}
T_n(x) &= \cos(n\arccos(x)) \\
&= \cos(-in\ln(x + \sqrt{x^2 - 1})) \\
&= \frac{1}{2} \Big( e^{i(-in\ln(x + \sqrt{x^2 - 1}))} + e^{-i(-in\ln(x + \sqrt{x^2 - 1}))} \Big) \\
&= \frac{1}{2} \Big( (x + \sqrt{x^2 - 1})^n + (x + \sqrt{x^2 - 1})^{-n} \Big) \\
&= \frac{1}{2} \Bigg( (x + \sqrt{x^2 - 1})^n + \frac{1}{(x + \sqrt{x^2 - 1})^n} \cdot \frac{(x - \sqrt{x^2 - 1})^n}{(x - \sqrt{x^2 - 1})^n} \Bigg) \\
&= \frac{1}{2} \Big( (x + \sqrt{x^2 - 1})^n + (x - \sqrt{x^2 - 1})^n \Big)
\end{split}
\end{equation*}
\end{proof}

\newpage
\begin{center}
\fbox{
\begin{minipage}{7.3 in}
\begin{proposition}[Extending the Chebyshev Polynomials] 
The definition of the \textit{Chebyshev Polynomials of the First Kind} can be extended for all real values:
\begin{equation*}
T_n(x) = \begin{cases}
(-1)^n \cosh(n\arccosh(-x)) & x \leq -1 \\
\cos(n\arccos(x)) & |x| < 1 \\
\cosh(n\arccosh(x)) & x \geq 1
\end{cases}
\end{equation*}
\end{proposition}
\end{minipage}}
\end{center}

\begin{proof}
The middle case is the one we already know. It has already been proven that hyperbolic cosine and its inverse can be used equivalently. Since the hyperbolic functions have a bigger domain, we let it define the region beyond $x = 1$. Now for all values below $x = -1$, consider the following:
\begin{equation*}
\begin{split}
T_n(x) &= \frac{1}{2} \Big( (x + \sqrt{x^2 - 1})^n + (x - \sqrt{x^2 - 1})^n \Big) \\
T_n(-x) &= \frac{1}{2} \Big( (-x + \sqrt{x^2 - 1})^n + (-x - \sqrt{x^2 - 1})^n \Big) \\
&= \frac{1}{2} \Big( (-1)^n(x - \sqrt{x^2 - 1})^n + (-1)^n(x + \sqrt{x^2 - 1})^n \Big) \\
&= (-1)^nT_n(x)
\end{split}
\end{equation*}
which tells us that the polynomials for negative values will either be exactly the same or off by a sign. With this we can write the extension below $x = -1$ the same as the other one with taking into account the sign:
\begin{equation*}
\begin{split}
T_n(x) &= (-1)^nT_n(-x) \\
&= (-1)^n\cosh(n\arccosh(-x))
\end{split}
\end{equation*}
for $x \leq -1$.
\end{proof}

\begin{center}
\fbox{
\begin{minipage}{7.3 in}
\begin{proposition}[Recurrence Definition] 
The \textit{Chebyshev Polynomials of the First Kind} can be calculated via the relations:
\begin{equation*}
\begin{split}
T_0(x) &= 1 \\
T_1(x) &= x \\
T_{n + 1}(x) &= 2xT_n(x) - T_{n - 1}(x)
\end{split}
\end{equation*}
\end{proposition}
\end{minipage}}
\end{center}

\begin{proof}
First check the two initial conditions:
\begin{equation*}
\begin{split}
T_0(x) &= \cos(0\cdot\arccos(x)) = 1 \\
T_1(x) &= \cos(1\cdot\arccos(x)) = x
\end{split}
\end{equation*}
Now for the recurrence relation we have:
\begin{equation*}
\begin{split}
T_{n + 1}(x) &= 2xT_n(x) - T_{n - 1}(x) \\
2xT_n(x) &= T_{n + 1}(x) + T_{n - 1}(x) \\
&= \cos((n + 1)\arccos(x)) + \cos((n - 1)\arccos(x)) \\
&= 2\cos\Big( \frac{(n + 1) + (n - 1)}{2} \arccos(x) \Big)\cos\Big( \frac{(n + 1) - (n - 1)}{2} \arccos(x) \Big) \\
&= 2x\cos(n\arccos(x)) \\
&= 2xT_n(x)
\end{split}
\end{equation*}
\end{proof}

\begin{center}
\fbox{
\begin{minipage}{7.3 in}
\begin{proposition}[Orthogonality] 
The \textit{Chebyshev Polynomials of the First Kind} form an orthogonal sequence of polynomials:
\begin{equation*}
\int_{-1}^1 T_n(x)T_m(x)w(x) dx = \begin{cases}
0 & n \neq m \\
\pi & n = m = 0 \\
\frac{\pi}{2} & n = m \neq 0
\end{cases}
\end{equation*}
with respect to the weight:
\begin{equation*}
w(x) = \frac{1}{\sqrt{1 - x^2}}
\end{equation*}
\end{proposition}
\end{minipage}}
\end{center}

\begin{proof}
To verify the orthogonality condition we make the substitution:
\begin{equation*}
x = \cos(\theta) \hspace{.2cm} \text{and} \hspace{.2cm} dx = -\sin(\theta) d\theta
\end{equation*}
and plug into the integral assuming that $n \neq m$:
\begin{equation*}
\begin{split}
\int_{-1}^1 T_n(x)T_m(x) \frac{dx}{\sqrt{1 - x^2}} &= \int_{-\pi}^0 T_n(\cos(\theta))T_m(\cos(\theta)) \frac{-\sin(\theta)}{\sqrt{1 - \cos^2(\theta)}} d\theta \\
&= \int_{-\pi}^0 T_n(\cos(\theta))T_m(\cos(\theta)) d\theta \\
&= \int_{-\pi}^0 \cos(n\theta)\cos(m\theta) d\theta \\
&= \frac{1}{2}\int_{-\pi}^0 (\cos((n - m)\theta) + \cos((n + m)\theta)) d\theta \\
&=  \frac{1}{2} \Big[ \frac{\sin((n - m)\theta)}{n - m} + \frac{\sin((n + m)\theta)}{n + m} \Big] \Big|_{-\pi}^0 \\
&= \frac{1}{2} \Big[ \frac{\sin((n - m)\pi)}{n - m} + \frac{\sin((n + m)\pi)}{n + m} \Big] \\
&= 0
\end{split}
\end{equation*}
Now for the case where $n = m \neq 0$:
\begin{equation*}
\begin{split}
\int_{-1}^1 T_n(x)T_m(x) \frac{dx}{\sqrt{1 - x^2}} &= \int_{-\pi}^0 T_n^2(\cos(\theta)) \frac{-\sin(\theta)}{\sqrt{1 - \cos^2(\theta)}} d\theta \\
&= \int_{-\pi}^0 T_n^2(\cos(\theta)) d\theta \\
&= \int_{-\pi}^0 \cos^2(n\theta) d\theta \\
&= \frac{1}{2} \int_{-\pi}^0 (\cos(2n\theta) + 1) d\theta \\
&= \frac{1}{2} \Big[ \frac{\sin(2n\theta)}{2n} + \theta \Big] \Big|_{-\pi}^0 \\
&= \frac{1}{2} \Big[ \frac{\sin(2n\pi)}{2n} + \pi \Big] \\
&= \frac{\pi}{2}
\end{split}
\end{equation*}
and finally for the case where $n = m = 0$:
\begin{equation*}
\begin{split}
\int_{-1}^1 T_n(x)T_m(x) \frac{dx}{\sqrt{1 - x^2}} &= \int_{-\pi}^0 T_0^2(\cos(\theta)) \frac{-\sin(\theta)}{\sqrt{1 - \cos^2(\theta)}} d\theta \\
&= \int_{-\pi}^0 T_0^2(\cos(\theta)) d\theta \\
&= \int_{-\pi}^0 d\theta \\
&= \theta \Big|_{-\pi}^0 \\
&= \pi
\end{split}
\end{equation*}
\end{proof}

\begin{center}
\fbox{
\begin{minipage}{7.3 in}
\begin{proposition}[Chebyshev Series] 
Since the \textit{Chebyshev Polynomials of the First Kind} form an orthogonal basis, they can be used to expand a function for $x \in [-1,1]$ as:
\begin{equation*}
f(x) = \sum_{n = 0}^\infty a_nT_n(x)
\end{equation*}
where:
\begin{equation*}
\begin{split}
a_0 &= \frac{2}{\pi} \int_{-1}^1 \frac{f(x)}{\sqrt{1 - x^2}} dx \\
a_n &= \frac{1}{\pi} \int_{-1}^1 \frac{T_n(x)f(x)}{\sqrt{1 - x^2}} dx
\end{split}
\end{equation*}
\end{proposition}
\end{minipage}}
\end{center}

\begin{proof}
We determine the coefficients by abusing the orthogonality of the basis polynomials:
\begin{equation*}
\begin{split}
f(x) &= \sum_{n = 0}^\infty a_nT_n(x) \\
\frac{T_m(x)f(x)}{\sqrt{1 - x^2}} &= \sum_{n = 0}^\infty a_n \frac{T_n(x)T_m(x)}{\sqrt{1 - x^2}} \\
\int_{-1}^1 \frac{T_m(x)f(x)}{\sqrt{1 - x^2}} dx &= \int_{-1}^1 \sum_{n = 0}^\infty a_n \frac{T_n(x)T_m(x)}{\sqrt{1 - x^2}} dx = \sum_{n = 0}^\infty a_n \int_{-1}^1 \frac{T_n(x)T_m(x)}{\sqrt{1 - x^2}} dx
\end{split}
\end{equation*}
where the integral and sum are allowed to switch order if we are staying inside the region of convergence. Now if $m \neq n$ the term goes to zero, thereby leaving only the terms where $n = m$. For this last piece there are two cases to consider. First if $m = n = 0$:
\begin{equation*}
\begin{split}
\frac{\pi}{2}a_0 &= \int_{-1}^1 \frac{T_0(x)f(x)}{\sqrt{1 - x^2}} dx \\
a_0 &= \frac{2}{\pi} \int_{-1}^1 \frac{f(x)}{\sqrt{1 - x^2}} dx
\end{split}
\end{equation*}
and for the rest:
\begin{equation*}
\begin{split}
\pi a_m &= \int_{-1}^1 \frac{T_m(x)f(x)}{\sqrt{1 - x^2}} dx \\
a_m &= \frac{1}{\pi} \int_{-1}^1 \frac{T_m(x)f(x)}{\sqrt{1 - x^2}} dx
\end{split}
\end{equation*}
\end{proof}

\begin{center}
\fbox{
\begin{minipage}{7.3 in}
\begin{definition}[Chebyshev Equation] 
The \textit{Chebyshev Equation} is a homogeneous second order linear differential equation in the form:
\begin{equation*}
(1 - x^2)\frac{d^2y}{dx^2} - x\frac{dy}{dx} + p^2y = 0
\end{equation*}
where $p \in \mathbb{R}$.
\end{definition}
\end{minipage}}
\end{center}

\begin{center}
\fbox{
\begin{minipage}{7.3 in}
\begin{proposition}[Solutions to the Chebyshev Equation] 
Assuming that the solution takes on a power series form:
\begin{equation*}
y(x) = \sum_{n = 0}^\infty a_n x^n
\end{equation*}
we have two common situations:
\begin{itemize}

\item[(1)]
Setting $a_0 = 1$ and $a_1 = 0$:
\begin{equation*}
y(x) = F(x) = 1 - \frac{p^2}{2!}x^2 + \frac{(p - 2)p^2(p + 2)}{4!}x^4 - \frac{(p - 4)(p - 2)p^2(p + 2)(p + 4)}{6!}x^6 + \dots
\end{equation*}

\item[(2)]
Setting $a_0 = 0$ and $a_1 = 1$:
\begin{equation*}
y(x) = G(x) = x - \frac{(p - 1)(p + 1)}{3!}x^3 + \frac{(p - 3)(p - 1)(p + 1)(p + 3)}{5!}x^5 - \dots
\end{equation*}

\end{itemize}
where the general solution is the linear combination of the two cases above.
\end{proposition}
\end{minipage}}
\end{center}

\begin{proof}
To check these solutions first we assume that the solution takes on a power series form and define:
\begin{equation*}
\begin{split}
y &= \sum_{n = 0}^\infty a_nx^n \\
y' &= \sum_{n = 1}^\infty a_nnx^{n - 1} \\
y'' &= \sum_{n = 2}^\infty a_nn(n - 1)x^{n - 2}
\end{split}
\end{equation*}
Now plugging into the differential equation:
\begin{equation*}
\begin{split}
0 &= (1 - x^2)\frac{d^2y}{dx^2} - x\frac{dy}{dx} + p^2y \\
&= (1 - x^2)\sum_{n = 2}^\infty a_nn(n - 1)x^{n - 2} - x\sum_{n = 1}^\infty a_nnx^{n - 1} + p^2\sum_{n = 0}^\infty a_nx^n \\
&= \sum_{n = 2}^\infty a_nn(n - 1)x^{n - 2} - \sum_{n = 2}^\infty a_nn(n - 1)x^n - \sum_{n = 1}^\infty a_nnx^n + p^2\sum_{n = 0}^\infty a_nx^n \\
&= \sum_{n = 0}^\infty a_{n + 2}(n + 2)(n + 1)x^n - \sum_{n = 2}^\infty a_nn(n - 1)x^n - a_1x - \sum_{n = 2}^\infty a_nnx^n + p^2a_0 + p^2a_1x + p^2\sum_{n = 2}^\infty a_nx^n \\
&= (2a_2 + p^2a_0) + (6a_3 + (p^2 - 1)a_1)x + \sum_{n = 2}^\infty \Big[ a_{n + 2}(n + 2)(n + 1) - a_n(n)(n - 1) - a_n(n) + p^2a_n \Big] x^n
\end{split}
\end{equation*}
Now if this polynomial is to equate to zero, each coefficient must simultaneously be zero giving:
\begin{equation*}
\begin{split}
a_2 &= -\frac{p^2}{2}a_0 \\
a_3 &= \frac{1 - p^2}{6}a_1 \\
a_{n + 2} &= -\frac{(p - n)(p + n)}{(n + 2)(n + 1)}a_n
\end{split}
\end{equation*}
Now if we set $a_0 = 1$ and $a_1 = 0$ we get:
\begin{equation*}
\begin{split}
a_0 &= 1 \\
a_1 &= 0 \\
a_2 &= -\frac{p^2}{2} \\
a_3 &= 0 \\
a_4 &= \frac{(p - 2)p^2(p + 2)}{4!} \\
&\vdots \\
a_{2n + 1} &= 0 \\
a_{2n} &= \frac{(-1)^np^2}{(2n)!}\prod_{i = 1}^{n - 1} (p - 2(n - i))(p + 2(n - i))
\end{split}
\end{equation*}
With this the solution takes on the form:
\begin{equation*}
\begin{split}
y(x) &= \sum_{n = 0}^\infty a_{2n}x^{2n} \\
&= 1 - \frac{p^2}{2}x^2 + \sum_{n = 2}^\infty \frac{(-1)^np^2}{(2n)!} \Bigg[ \prod_{i = 1}^{n - 1} (p - 2(n - i))(p + 2(n - i)) \Bigg] x^{2n} \\
&= F(x)
\end{split}
\end{equation*}
Now if we set $a_0 = 0$ and $a_1 = 1$ we get:
\begin{equation*}
\begin{split}
a_0 &= 0 \\
a_1 &= 1 \\
a_2 &= 0 \\
a_3 &= -\frac{(p - 1)(p + 1)}{3!} \\
a_4 &= 0 \\
&\vdots \\
a_{2n + 1} &= \frac{(-1)^n}{(2n + 1)!}\prod_{i = 1}^{n - 1} (p - (2n + 1 - 2i))(p + (2n + 1 - 2i)) \\
a_{2n} &= 0
\end{split}
\end{equation*}
With this the solution takes on the form:
\begin{equation*}
\begin{split}
y(x) &= \sum_{n = 0}^\infty a_{2n + 1}x^{2n + 1} \\
&= x - \frac{(p - 1)(p + 1)}{3!}x^3 + \sum_{n = 2}^\infty \frac{(-1)^n}{(2n + 1)!} \Bigg[ \prod_{i = 1}^{n - 1} (p - (2n + 1 - 2i))(p + (2n + 1 - 2i)) \Bigg] x^{2n + 1} \\
&= G(x)
\end{split}
\end{equation*}
Note that each of these solutions has the missing components of the other, therefore we form the general solution as the linear combination of the two.
\end{proof}

\newpage
\begin{center}
\fbox{
\begin{minipage}{7.3 in}
\begin{proposition}[Convergence of the Solutions] 
The power series solution to the Chebyshev equation converges iff $|x| \leq 1$.
\end{proposition}
\end{minipage}}
\end{center}

\begin{proof}
We begin by determining the limit:
\begin{equation*}
\lim_{n \rightarrow \infty} \Big| \frac{a_{n + 2}}{a_n} \Big| = |x^2| \lim_{n \rightarrow \infty} \Bigg| \frac{(p - n)(p + n)}{(n + 2)(n + 1)} \Bigg| = |x|^2
\end{equation*}
and now by the ratio test:
\begin{equation*}
|x|^2 < 1 \implies |x| < 1
\end{equation*}
With this we have the region and have to check the boundaries independently. So for $x = -1$ we have:
\begin{equation*}
\begin{split}
y &= \frac{(p - 1)(p + 1)}{3!} + \sum_{n = 2}^\infty \frac{(-1)^{n + 1}}{(2n + 1)!} \Bigg[ \prod_{i = 1}^{n - 1} (p - (2n + 1 - 2i))(p + (2n + 1 - 2i)) \Bigg] \\
&\hspace{.4cm} - \frac{p^2}{2} + \sum_{n = 2}^\infty \frac{(-1)^np^2}{(2n)!} \Bigg[ \prod_{i = 1}^{n - 1} (p - 2(n - i))(p + 2(n - i)) \Bigg]
\end{split}
\end{equation*}
where both infinite series converge according to the alternating series test. Lastly for $x = 1$:
\begin{equation*}
\begin{split}
y &= 1 - \frac{(p - 1)(p + 1)}{3!} + \sum_{n = 2}^\infty \frac{(-1)^n}{(2n + 1)!} \Bigg[ \prod_{i = 1}^{n - 1} (p - (2n + 1 - 2i))(p + (2n + 1 - 2i)) \Bigg] \\
&\hspace{.4cm} + 1 - \frac{p^2}{2} + \sum_{n = 2}^\infty \frac{(-1)^np^2}{(2n)!} \Bigg[ \prod_{i = 1}^{n - 1} (p - 2(n - i))(p + 2(n - i)) \Bigg]
\end{split}
\end{equation*}
and once again by the alternating series test both infinite series converge above. Therefore, the final region of convergence included the boundaries and is given by $[-1,1]$.
\end{proof}

\begin{center}
\fbox{
\begin{minipage}{7.3 in}
\begin{proposition}[Chebyshev Polynomials from the Chebyshev Equation] 
Notice that if from the above we have $p \in \mathbb{Z}$, the first and second case will terminate if $p$ is even and odd respectively. Therefore, the two cases reduce down to a $p$th degree polynomial that is proportional to the $p$th Chebyshev polynomial by the relations:
\begin{equation*}
T_p(x) = \begin{cases}
(-1)^{\frac{p}{2}} F(x) & \exists k \in \mathbb{N} \hspace{.2cm} \text{s.t.} \hspace{.2cm} p = 2k \\
(-1)^{\frac{p - 1}{2}}p G(x) & \exists k \in \mathbb{N} \hspace{.2cm} \text{s.t.} \hspace{.2cm} p = 2k + 1
\end{cases}
\end{equation*}
\end{proposition}
\end{minipage}}
\end{center}








\end{document}
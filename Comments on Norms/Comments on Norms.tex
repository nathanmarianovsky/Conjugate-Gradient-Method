\documentclass[12pt, letterpaper, onecolumn, conference, final]{IEEEtran}

\usepackage[margin = .5in]{geometry}
\usepackage{amsmath}
\usepackage{amsthm}
\usepackage{amssymb}
\usepackage{wasysym}
\usepackage{graphicx}
\usepackage[makeroom]{cancel}
\usepackage{polynom}
\usepackage{booktabs}

\title{Comments on Norms}
\author{Nathan Marianovsky}

\theoremstyle{definition}
\newtheorem{definition}{Definition}
\newtheorem{proposition}{Proposition}

\theoremstyle{plain}
\newtheorem{theorem}{Theorem}[section]
\newtheorem{example}{Example}
\newtheorem{solution}{Solution}

\renewcommand{\qedsymbol}{$\blacksquare$}

\renewcommand\thesection{\arabic{section}}

\begin{document}

\maketitle

\begin{center}
\fbox{
\begin{minipage}{7.3 in}
\begin{definition}[Norms] 
Given a vector space $V$ over some field $F$, the \textit{norm} is defined to be the function:
\begin{equation*}
\rho: V \rightarrow \mathbb{R}
\end{equation*}
s.t.:
\begin{itemize}

\item[(a)]
Given $\alpha \in F$ and $v \in V$:
\begin{equation*}
\rho(\alpha v) = |\alpha| \rho(v)
\end{equation*}

\item[(b)]
Given $v_1,v_2 \in V$ the triangle inequality holds:
\begin{equation*}
\rho(v_1 + v_2) \leq \rho(v_1) + \rho(v_2)
\end{equation*}

\item[(c)]
Given $v \in V$ the norm is non-negative:
\begin{equation*}
\rho(v) \geq 0
\end{equation*}
where equality holds iff $v = 0$.

\end{itemize}
Note that typically fields of interest include the real or complex numbers.
\end{definition}
\end{minipage}}
\end{center}

\begin{center}
\fbox{
\begin{minipage}{7.3 in}
\begin{proposition}[Euclidean Norm] 
Given the vector space $V = \mathbb{R}^n$ over the field $F = \mathbb{R}$, the Euclidean norm is defined as:
\begin{equation*}
\| v \| = \sqrt{v^Tv} = \sqrt{\sum v_i^2}
\end{equation*}
where $v \in V$.
\end{proposition}
\end{minipage}}
\end{center}

\begin{proof}
We just have to ensure that the three conditions for a norm hold true:
\begin{itemize}

\item[(a)]
\begin{equation*}
\begin{split}
\| v \| &= \sqrt{(\alpha v^T)(\alpha v)} \\
&= \sqrt{\alpha^2}\sqrt{v^Tv} \\
&= |\alpha|\sqrt{v^Tv}
\end{split}
\end{equation*}

\item[(b)]
Using the Cauchy-Schwarz Inequality:
\begin{equation*}
\begin{split}
\| v_1 + v_2 \|^2 &= \sum (v_{1i} + v_{2i})^2 \\
\| v_1 + v_2 \|^2 &= \sum v_{1i}^2 + \sum v_{2i}^2 + 2\sum v_{1i}v_{2i} \\
\| v_1 + v_2 \|^2 &= \| v_1 \|^2 + \| v_2 \|^2 + 2\sum v_{1i}v_{2i} \\
\| v_1 + v_2 \|^2 &\leq \| v_1 \|^2 + \| v_2 \|^2 + 2\sqrt{\sum v_{1i}^2}\sqrt{\sum v_{2i}^2} \\
\| v_1 + v_2 \|^2 &\leq \| v_1 \|^2 + \| v_2 \|^2 + 2\| v_1 \| \| v_2 \| \\
\| v_1 + v_2 \|^2 &\leq (\| v_1 \| + \| v_2 \|)^2 \\
\| v_1 + v_2 \| &\leq \| v_1 \| + \| v_2 \|
\end{split}
\end{equation*}

\item[(c)]
For this consider the fact that the inside of the norm is a sum of strictly positive values. That sum will only be zero is each component is zero. As for the positive behavior consider the fact that:
\begin{equation*}
\sqrt{x} \geq 0
\end{equation*}
which confirms that norm is strictly positive and zero only if the sum is zero which implies all of the components are zero.

\end{itemize}
\end{proof}

\newpage
\begin{center}
\fbox{
\begin{minipage}{7.3 in}
\begin{proposition}[Energy Norm] 
Given the vector space $V = \mathbb{R}^n$ over the field $F = \mathbb{R}$, the energy norm is defined as:
\begin{equation*}
\| v \|_A = \sqrt{v^TAv}
\end{equation*}
where $v \in V$ and $A \in \mathbb{R}^{2n}$ where it is known to be positive-definite and symmetric.
\end{proposition}
\end{minipage}}
\end{center}

\begin{proof}
We just have to ensure that the three conditions for a norm hold true:
\begin{itemize}

\item[(a)]
\begin{equation*}
\begin{split}
\| v \|_A &= \sqrt{(\alpha v^T)A(\alpha v)} \\
&= \sqrt{\alpha^2}\sqrt{v^TAv} \\
&= |\alpha|\sqrt{v^TAv}
\end{split}
\end{equation*}

\item[(b)]
Using the Cauchy-Schwarz Inequality and the fact that $A$ is a symmetric matrix:
\begin{equation*}
\begin{split}
\| v_1 + v_2 \|_A^2 &= (v_1 + v_2)^TA(v_1 + v_2) \\
\| v_1 + v_2 \|_A^2 &= (v_1^T + v_2^T)(Av_1 + Av_2) \\
\| v_1 + v_2 \|_A^2 &= v_1^TAv_1 + v_2^TAv_2 + v_1^TAv_2 + v_2^TAv_1 \\
\| v_1 + v_2 \|_A^2 &= \| v_1 \|_A^2 +  \| v_2 \|_A^2 + 2v_2^TAv_1 \\
\| v_1 + v_2 \|_A^2 &= \| v_1 \|_A^2 +  \| v_2 \|_A^2 + 2v_2^TA^\frac{1}{2}A^\frac{1}{2}v_1 \\
\| v_1 + v_2 \|_A^2 &= \| v_1 \|_A^2 +  \| v_2 \|_A^2 + 2((A^\frac{1}{2})^Tv_2)^T(A^\frac{1}{2}v_1) \\
\| v_1 + v_2 \|_A^2 &\leq \| v_1 \|_A^2 +  \| v_2 \|_A^2 + \| (A^\frac{1}{2})^T v_2 \| \| A^\frac{1}{2} v_1 \| \\
\| v_1 + v_2 \|_A^2 &\leq \| v_1 \|_A^2 +  \| v_2 \|_A^2 + \| A^\frac{1}{2} v_2 \| \| A^\frac{1}{2} v_1 \| \\
\| v_1 + v_2 \|_A^2 &\leq \| v_1 \|_A^2 +  \| v_2 \|_A^2 + \| v_2 \|_A \| v_1 \|_A \\
\| v_1 + v_2 \|_A^2 &\leq (\| v_1 \|_A + \| v_2 \|_A)^2 \\
\| v_1 + v_2 \|_A &\leq \| v_1 \|_A + \| v_2 \|_A 
\end{split}
\end{equation*}

\item[(c)]
For this consider that if $A$ is positive-definite, then:
\begin{equation*}
v^TAv \geq 0
\end{equation*}
with equality only when $v = 0$. So the square guarantees that we only take on positive values where zero is achieved iff the vector is identically the zero vector.

\end{itemize}
\end{proof}



\end{document}
